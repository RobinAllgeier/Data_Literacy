%%%%%%%% DATA LITERACY 2025 LATEX PROJECT TEMPLATE FILE %%%%%%%%%%%%%%%%%
%%% Based on the 2025 ICML template, available at https://icml.cc/Conferences/2025/AuthorInstructions %%%

\documentclass{article}

% Recommended, but optional, packages for figures and better typesetting:
\usepackage{microtype}
\usepackage{graphicx}
\usepackage{subfigure}
\usepackage{booktabs} % for professional tables

\usepackage{tikz}
% Corporate Design of the University of Tübingen
% Primary Colors
\definecolor{TUred}{RGB}{165,30,55}
\definecolor{TUgold}{RGB}{180,160,105}
\definecolor{TUdark}{RGB}{50,65,75}
\definecolor{TUgray}{RGB}{175,179,183}

% Secondary Colors
\definecolor{TUdarkblue}{RGB}{65,90,140}
\definecolor{TUblue}{RGB}{0,105,170}
\definecolor{TUlightblue}{RGB}{80,170,200}
\definecolor{TUlightgreen}{RGB}{130,185,160}
\definecolor{TUgreen}{RGB}{125,165,75}
\definecolor{TUdarkgreen}{RGB}{50,110,30}
\definecolor{TUocre}{RGB}{200,80,60}
\definecolor{TUviolet}{RGB}{175,110,150}
\definecolor{TUmauve}{RGB}{180,160,150}
\definecolor{TUbeige}{RGB}{215,180,105}
\definecolor{TUorange}{RGB}{210,150,0}
\definecolor{TUbrown}{RGB}{145,105,70}

\definecolor{pn_orange}{RGB}{230,97,0}

% hyperref makes hyperlinks in the resulting PDF.
% If your build breaks (sometimes temporarily if a hyperlink spans a page)
% please comment out the following usepackage line and replace
% \usepackage{icml2023} with \usepackage[nohyperref]{icml2023} above.
\usepackage{hyperref}


% Attempt to make hyperref and algorithmic work together better:
\newcommand{\theHalgorithm}{\arabic{algorithm}}

\usepackage[accepted]{icml2025}

% For theorems and such
\usepackage{amsmath}
\usepackage{amssymb}
\usepackage{mathtools}
\usepackage{amsthm}

% if you use cleveref..
\usepackage[capitalize,noabbrev]{cleveref}

% Todonotes is useful during development; simply uncomment the next line
%    and comment out the line below the next line to turn off comments
%\usepackage[disable,textsize=tiny]{todonotes}
\usepackage[textsize=tiny]{todonotes}


% The \icmltitle you define below is probably too long as a header.
% Therefore, a short form for the running title is supplied here:
\icmltitlerunning{Project Report for Data Literacy 2025}

\begin{document}

\twocolumn[
\icmltitle{Learning and Habits: The Borrowing Behavior of Library Users}

% It is OKAY to include author information, even for blind
% submissions: the style file will automatically remove it for you
% unless you've provided the [accepted] option to the icml2023
% package.

% List of affiliations: The first argument should be a (short)
% identifier you will use later to specify author affiliations
% Academic affiliations should list Department, University, City, Region, Country
% Industry affiliations should list Company, City, Region, Country

% You can specify symbols, otherwise they are numbered in order.
% Ideally, you should not use this facility. Affiliations will be numbered
% in order of appearance and this is the preferred way.
\icmlsetsymbol{equal}{*}

\begin{icmlauthorlist}
\icmlauthor{Nico Rinck}{equal,first}
\icmlauthor{Adriano Polzer}{equal,second}
\icmlauthor{Robin Allgeier}{equal,third}
\icmlauthor{Jonas Mahr}{equal,fourth}
\icmlauthor{Jannik Rombach}{equal,fifth}
\end{icmlauthorlist}

% fill in your matrikelnummer, email address, degree, for each group member
\icmlaffiliation{first}{Matrikelnummer 7314094, MSc Computer Science}
\icmlaffiliation{second}{Matrikelnummer 7366709, MSc Computer Science}
\icmlaffiliation{third}{Matrikelnummer 6288873, MSc Computer Science}
\icmlaffiliation{fourth}{Matrikelnummer 7322288, MSc Computer Science}
\icmlaffiliation{fifth}{Matrikelnummer 7317181, MSc Computer Science}


% put your email addresses here. You can use initials to save space,
% e.g. if you are called Max Mustermann, you can use \icmlcorrespondingauthor{MM}{max.mustermann@uni-tuebingen.de}
% DO USE YOUR UNIVERSITY EMAIL ADDRESS!

% for the Data Literacy report, to save space, you can here list the student email address of one author, who is willing to be contacted about this work in the future (e.g. in case we would like to use your report as an example for future course iterations)
%\icmlcorrespondingauthor{NR}{nico.rinck@uni-tuebingen.de} 
%\icmlcorrespondingauthor{AP}{adriano.polzer@uni-tuebingen.de}
%\icmlcorrespondingauthor{RA}{robin.allgeier@uni-tuebingen.de}
%\icmlcorrespondingauthor{JM}{jonas.mahr@uni-tuebingen.de}
>>>>>>> dda3ad6 (fix template)
\icmlcorrespondingauthor{JR}{jannik.rombach@uni-tuebingen.de}

% You may provide any keywords that you
% find helpful for describing your paper; these are used to populate
% the "keywords" metadata in the PDF but will not be shown in the document
%\icmlkeywords{Machine Learning, ICML}

\vskip 0.3in
]

% this must go after the closing bracket ] following \twocolumn[ ...

% This command actually creates the footnote in the first column
% listing the affiliations and the copyright notice.
% The command takes one argument, which is text to display at the start of the footnote.
% The \icmlEqualContribution command is standard text for equal contribution.
% Remove it (just {}) if you do not need this facility.

%\printAffiliationsAndNotice{}  % leave blank if no need to mention equal contribution
\printAffiliationsAndNotice{\icmlEqualContribution} % otherwise use the standard text.

\begin{abstract}
	Late returns of library materials disrupt circulation efficiency and strain administrative resources. This study analyzes 2.4 million loan transactions from approximately 21,000 users at the Tübingen City Library (2019–2025) to understand borrowing behavior patterns and learning effects. We develop a session-based analytical framework that aggregates item-level records into user visits. Our analysis reveals that late return probability declines sharply during early sessions, while extension requests increase from 50\% to 65\%, indicating proactive learning rather than simple compliance. Borrowing activity exhibits strong temporal patterns aligned with operating hours, and media-type preferences remain influential but become less predictive with experience. These findings suggest that targeted communication systems for new users may be more effective than uniform enforcement mechanisms.
\end{abstract}

\section{Introduction}\label{sec:intro}

% Motivate the problem, situation or topic you decided to work on. Describe why it matters (is it of societal, economic, scientific value?). Outline the rest of the paper (use references, e.g.~to \Cref{sec:methods}: What kind of data you are working with, how you analyse it, and what kind of conclusion you reached. The point of the introduction is to make the reader want to read the rest of the paper.

Public libraries play an important role in providing people with access to knowledge, education, and entertainment. To maintain efficient operations and ensure equitable resource availability, libraries must understand how their users interact with the collection. Analyzing borrowing behavior provides valuable insights into usage patterns, user preferences, and operational challenges, enabling data-driven decision-making to improve service quality.
Understanding borrowing behavior encompasses multiple dimensions: which types of media are most frequently borrowed, how different user groups interact with the library, how loan durations and extension patterns vary across media types and user categories, and temporal trends in borrowing activity. One particularly important aspect of borrowing behavior is the timely return of borrowed items. Late returns can lead to reduced availability of popular materials, diminished satisfaction among users waiting for specific items, and increased administrative burdens for library staff. Identifying factors associated with late returns enables libraries to adapt their policies, such as loan duration limits, extension rules, or targeted reminder systems—to improve resource circulation and user satisfaction.
This study analyzes borrowing records from a public library. The primary objective is to characterize borrowing behavior across multiple dimensions and to identify patterns that contribute to our understanding of library usage. Specifically, this study addresses the following questions: \textit{How do borrowing patterns vary across media types, user categories, and time periods? What observable characteristics of a loan are associated with late returns?} The analysis investigates temporal patterns, in borrowing activity, and examines the relationships between late returns and several factors.

Our analysis reveals several clear patterns: \dots
\section{Data and Methods}\label{sec:methods}

%In this section, describe \emph{what you did}. Roughly speaking, explain what data you worked with, how or from where it was collected, it's structure and size. Explain your analysis, and any specific choices you made in it. Depending on the nature of your project, you may focus more or less on certain aspects. If you collected data yourself, explain the collection process in detail. If you downloaded data from the net, show an exploratory analysis that builds intuition for the data, and shows that you know the data well. If you are doing a custom analysis, explain how it works and why it is the right choice. If you are using a standard tool, it may still help to briefly outline it. Cite relevant works. You can use the \verb|\citep| and \verb|\citet| commands for this purpose \citep{mackay2003information}.

% This is the template for a figure from the original ICML submission pack. In lecture 10 we will discuss plotting in detail.
% Refer to this lecture on how to include figures in this text.
% 
% \begin{figure}[ht]
% \vskip 0.2in
% \begin{center}
% \centerline{\includegraphics[width=\columnwidth]{icml_numpapers}}
% \caption{Historical locations and number of accepted papers for International
% Machine Learning Conferences (ICML 1993 -- ICML 2008) and International
% Workshops on Machine Learning (ML 1988 -- ML 1992). At the time this figure was
% produced, the number of accepted papers for ICML 2008 was unknown and instead
% estimated.}
% \label{icml-historical}
% \end{center}
% \vskip -0.2in
% \end{figure}

\subsection{Data Collection and Description}
This study investigates the borrowing behavior of users at the Tübingen City Library, as well as the frequency of late returns. The dataset was provided directly by the Tübingen City Library and covers borrowing records from 2019 to 2025. It encompasses over 2.4 million individual loan transactions and includes the following key variables: borrowing and return timestamps, 21 different media types, user categories, number of extensions, anonymized user identification numbers, as well as late return flags. Each record documents a complete transaction from initial borrowing to return.
Domain knowledge was gathered through a personal interview with a staff member of the Tübingen City Library, providing contextual insights into the data collection processes, library operations, and data quality practices. Additional information was obtained through ongoing email communication with the library staff, which helped to clarify data inconsistencies and provide domain expertise for informed analysis decisions.

\subsection{Data Quality Assessment: Sanity Checks}
Prior to conducting any analysis, a comprehensive set of data quality checks was performed to validate internal consistency and identify potential data quality issues. These sanity checks were designed to detect anomalies without filtering or removing data at first, and gather information, regarding cleaning the data to ensure robust analysis afterwards. The following specific checks were implemented:

\subsubsection{Missing Values}
First of all, the overall of missing values in the dataset was assessed. The proportion of missing values in each column was calculated to identify any fields with significant gaps that could impact the analysis. Depending on the extent and nature of the missing valus, they were excluded from certain analyses.
Return timestamps are missing in roughly two percent of rows, which is expected for currently borrowed and unreturned or lost items and therefore retained. The same rate applies to the derived duration column because it depends on return timestamps. A missing rate of about 6.7\% was found in the user ID colum, which will be handled accordingly in the analysis. All other colums, which are relevant for our analyses, have a missing rate below 0.1\% and are therefore considered complete.

\subsubsection{Timestamp and Duration Consistency}
The integrity of temporal data was verified by identifying logical inconsistencies, such as instances where return dates preceded borrowing dates. These checks are critical since all downstream analyses depend on accurate temporal information. Further checks on the duration value revealed that it is calculated correctly from the timestamps and does not contain any inconsistencies.

\subsubsection{Late Return Consistency}
The consistency between late return flags and the reported number of days late was validated. No Entries were identified where the late flag indicated ``not late'' but the days-late counter was positive, and conversely, records marked as late but with zero or missing days-late values. Additionally, implausible values in the extensions column (negative extensions or more than six extensions) were checked. There are a few cases where the number of extensions is higher than six, which should not be possible according to the library's policies. But because of the low number of affected rows (0.003\%), these where not further investigated.

\subsubsection{Duplicate Analysis}
Various forms of duplicates were examined. There are no exact duplicates in the dataset. However, some entries show the the exact same borrowing timestamp. These can be explained by the fact, that users can borrow at multiple stations at the same time. Also it is possible for user to borrow multiple items in one transaction, resulting in identical borrowing timestamps for different items, but all these occurrences limit to the same user ID and to a maximum of seven items/entries per transaction. Only one case was found where a different user borrowed and returned at the exact same times, which is considered a rare but possible event and therefore retained.

\subsection{Data cleaning}
Based on the result of the data quality assessment and the analysis approach defined above, the data was cleaned, based on four main rules:
\begin{itemize}
    \item Entries with missing retuern timestamps were retained, as they represent currently borrowed or lost items.
    \item Entries with a borrowing time longer than the 196 days were removed, because they extend the maximum borrowing period of 28 days and six extentions, which are considered implausible.
    \item Borrowings taken by users with the user type MDA and SYS were removed, because these represent borrowings from the library`s staff members and system entries, which do not reflect typical user behavior.
    \item For some analyses, all media types exept for books were removed, because they have different borrowing duration limits and would distort the results.
\end{itemize}
These cleaning steps ensured that the dataset used for analysis was both reliable and relevant, thereby enhancing the validity of the findings.


\subsection{Analysis Methods}

To move from item-level borrowing records to user-level behavioral analysis, individual borrowings were aggregated into user-specific sessions. 
Each record corresponds to a single item, while users often borrow multiple items during one library visit. 
A session is therefore defined as all borrowings by the same user on the same calendar day. 
Sessions were ordered chronologically per user and assigned a session index, which serves as a proxy for increasing experience. 
Session-level indicators were derived by aggregation. A session was marked as late if any item was returned after the due date and as extended if any item received a loan extension. This session-based representation forms the common basis for all subsequent analyses.

Based on this representation, temporal usage patterns and visit regularity were quantified using borrowing timestamps. 
Borrowing events were aggregated by weekday and hour, and the number of active users was computed in hourly bins to summarize population-level usage. 
Individual visit regularity was measured by defining a typical borrowing time for each user as the modal weekday and hour across their borrowing history. 
Each event was labeled according to whether it matched this typical time. Aggregating this indicator across users yields a population-level measure of regularity.

Behavioral adaptation over time was analyzed using the session index as a measure of user experience. 
For each experience level $k$, we computed the proportion of late sessions and sessions with extensions across all users who reached at least $k$ sessions. 
This ensures that each user contributes at most one observation per level. Let $L_{u,k}$ denote whether the $k$-th session of user $u$ contains at least one late item. 
The late-return learning curve is
\[
\hat{p}_{L}(k) = \frac{1}{|U_k|} \sum_{u \in U_k} L_{u,k},
\]
where $U_k$ denotes the set of users with at least $k$ sessions. Extensions were computed analogously.

Uncertainty was estimated using user-level bootstrap resampling. 
Because observations are dependent within users and the sampling distribution of $\hat{p}_L(k)$ is unknown, parametric methods are inappropriate. 
Users were therefore resampled with replacement, session sequences were reconstructed, and learning curves recomputed for each bootstrap sample. 
Confidence intervals were obtained from the resulting empirical distributions \cite{Davison_Hinkley_1997}.

\section{Results}\label{sec:results}

%In this section outline your results. At this point, you are just stating the outcome of your analysis. You can highlight important aspects (``we observe a significantly higher value of $x$ over $y$''), but leave interpretation and opinion to the next section. This section absoultely \emph{must} include at least two figures.

Borrowing activity at the Tübingen City Library exhibits strong temporal patterns tied to operating hours and user schedules.
\autoref{fig:visit-clock} displays average session counts aggregated into half-hour bins, separately for weekdays (Tuesday–Friday) and Saturdays (Monday/Sunday closed).
Each bin represents the mean number of sessions occurring within that time window, averaged across all recorded days.

\begin{figure}[ht]
    \centering
    \includegraphics{figures/plot_1_clock_plot.pdf}
    \caption{
        Average library usage by time of day on a 24-hour clock.
        Mean session counts per time bin are shown for Tue--Fri
        (\textcolor{TUblue}{%
            \tikz[baseline]{%
                \draw[fill=TUblue!25, draw=TUblue, line width=0.6pt]
                (0,0) rectangle (0.55em,0.55em);
            }
        }\!)
        and Saturday
        (\textcolor{pn_orange}{%
            \tikz[baseline]{%
                \draw[fill=pn_orange!25, draw=pn_orange, line width=0.6pt]
                (0,0) rectangle (0.55em,0.55em);
            }
        }\!).
        Nighttime hours are visually compressed to reflect library closing times, which are indicated by grey radial bands.
    }
    \label{fig:visit-clock}
\end{figure}

Usage concentrates heavily during daytime hours, with clear peaks from late morning through afternoon.
Weekdays show sustained high volume from opening through early afternoon, with a noticeable dip around midday consistent with lunch breaks.
Activity rises again in the afternoon, aligning with work and school schedules.
Saturday exhibits a later start and more compressed activity, suggesting leisure-driven rather than routine use.
These aggregate patterns reveal shared daily rhythms shaped by operational hours and typical schedules, though individuals vary considerably in their temporal regularity.

Among users with at least 10 sessions, the mean probability of returning on the same weekday is 40.4\%, indicating moderate day-of-week preferences, with some showing stronger consistency.
Visit times within a day are more dispersed, with a mean standard deviation of 2.1 hours across users, suggesting most do not restrict themselves to narrow time windows.
However, approximately 3\% exhibit high temporal precision, with standard deviations below 1 hour, enabling identification of preferred time slots based on their mean visit times.

Beyond these temporal patterns, user behavior evolves with repeated library engagement, revealing learning effects in late returns and loan extensions.
\autoref{fig:learning-curve} plots both probabilities against the session index, tracking users from their first visit through their 25th session.
Late-return probability shows a sharp decline over the first nine sessions, dropping from approximately 13\% to below 10\%, indicating rapid acquisition of library procedures such as due dates and reminder systems.
Beyond session 10, the curve stabilizes into a relatively flat plateau around 8–9\%, suggesting users have transitioned from active learning into stable behavioral routines.

\begin{figure}[ht]
    \centering
    \includegraphics{figures/plot_2_learning_curve.pdf}
    \caption{
        Session-based late-return (\textcolor{pn_orange}{%
            \tikz\draw[line width=1.2pt] (0,0) -- (0.9em,0)
            node[midway, circle, fill=pn_orange, inner sep=1.2pt] {};
        } left axis)
        and extension
        (\textcolor{TUblue}{%
            \tikz\draw[line width=1.2pt] (0,0) -- (0.9em,0)
            node[midway, circle, fill=TUblue, inner sep=1.2pt] {};
        } right axis) probabilities
        across the user session index.
        Shaded bands denote user-level bootstrap confidence intervals,
        while thin lines show raw estimates and thicker lines rolling averages.
    }
    \label{fig:learning-curve}
\end{figure}

Extension probability exhibits the inverse pattern, rising from approximately 50\% to over 65\% during the same experience window.
This increase reveals a shift in user strategy: rather than simply becoming more punctual, experienced users proactively request extensions to avoid lateness.
The diverging trajectories suggest that learning manifests not as perfect compliance, but as adaptive behavior within library constraints.
Notably, late returns never approach zero even among highly experienced users, indicating that some lateness stems from structural factors such as fixed loan limits or schedule conflicts, independent of user familiarity with procedures.

\begin{figure}[ht]
    \centering
    \includegraphics{figures/plot_4_media_type_stickiness.pdf}
    \caption{
        Consistency with early dominant media type across the user session index for
        $k_0=1$
        (\textcolor{pn_orange}{%
            \tikz\draw[line width=1.2pt] (0,0) -- (0.9em,0)
            node[midway, circle, fill=pn_orange, inner sep=1.2pt] {};
        }\!),
        $k_0=5$
        (\textcolor{TUblue}{%
            \tikz\draw[line width=1.2pt] (0,0) -- (0.9em,0)
            node[midway, circle, fill=TUblue, inner sep=1.2pt] {};
        }\!),
        and $k_0=10$
        (\textcolor{TUred}{%
            \tikz\draw[line width=1.2pt] (0,0) -- (0.9em,0)
            node[midway, circle, fill=TUred, inner sep=1.2pt] {};
        }\!).
        For each $k_0$, the curve starts at session $k_0+1$.
        Shaded bands denote user-level bootstrap confidence intervals,
        while thin lines show raw estimates and thicker lines rolling averages.
    }
    \label{fig:media-stickiness}
\end{figure}


In addition to timing and return behavior, we analyze the consistency of media-type choices over repeated sessions at the user level.
\autoref{fig:media-stickiness} shows, for $k_0 \in \{1,5,10\}$, the share of sessions whose dominant media type matches
the user’s early preferred type, plotted against the session index. Across all three curves, consistency decreases with
increasing session index, with the strongest decline for smaller $k_0$.
For $k_0=1$, users match their first-session dominant media type in about 64\% of second sessions.
By session 25, this value drops to roughly 50\%, a decrease of about 14 percentage points.
For $k_0=10$, the decrease is about 7 percentage points, and the curve remains close to 60\% in later sessions.
This is consistent with the fact that choosing a larger $k_0$ defines the early preferred type from more observations, making it more robust and less sensitive
to variability in the first sessions.
Overall, the curves flatten over time, indicating that most change in dominant media type occurs in the early sessions,
while behavior becomes more stable at higher session indices.

\section{Discussion \& Conclusion}\label{sec:conclusion}

% Use this section to briefly summarize the entire text. Highlight limitations and problems, but also make clear statements where they are possible and supported by the analysis. 
Public libraries generate rich, multifaceted datasets that provide a detailed view of user borrowing behavior. However, leveraging these data requires extensive domain knowledge. Understanding library rules, operating hours, internal system conventions, and similar operational details is essential to interpret the data correctly. This work was based on a non-public and mostly unprocessed dataset from the Tübingen City Library. It goes beyond the internal statistics compiled by the library. This setting presented both challenges and opportunities. 
A significant amount of the work was devoted to evaluating and cleaning up the data quality, which relied in parts on the library's insights to resolve inconsistencies and understand the structure of the data.
There are several limitations to consider. First, the dataset only includes borrowing records from 2019 onwards, as earlier data were collected under different system conventions. This restricts the temporal scope of the analysis and prevents the investigation of longer-term trends. Second, user behavior can only be observed indirectly through recorded transactions. As the temporal analysis suggests, several sources of bias and external influences are not captured in the data, such as constraints on user schedules, social contexts, or unobserved personal circumstances, which limit the precision with which behavioral patterns can be inferred. For example, it remains unknown whether items are borrowed for the users themselves or for others (e.g., parents borrowing for their children), which can affect the interpretation of observed behavior.
Nevertheless, the analysis of user behavior revealed both aggregated usage patterns describing daily library use and individual-level behaviors. The results indicate that temporally consistent borrowing behavior is difficult for the majority of users, while a small subgroup nevertheless exhibits highly regular temporal routines.
In addition, the data show learning effects with repeated use. This reflects not only improved handling of late returns, but also the development of adaptive strategies that reduce late returns through active use of the renewal system.

Overall, the results also have practical relevance for the library. Understanding user behavior can inform staffing decisions, opening hours, and the design of reminder systems or targeted communication. Moreover, the session-based representation and preprocessing pipeline provides a foundation for further analyses on this dataset.

\newpage

\section*{Contribution Statement}
Nico Rinck analyzed media types in relation to user behavior. Jonas Mahr examined users’ interest categories. Adriano Polzer analyzed user behavior with respect to time of day. Robin Allgeier obtained the data and prepared it through sanity checks and data cleaning. Jannik Rombach prepared the data and analyzed users’ learning behavior.

\bibliography{bibliography}
\bibliographystyle{icml2025}

\end{document}

% This document was modified from the files available at https://icml.cc/Conferences/2025/AuthorInstructions
% the full copyright notice is available within the file icml2025.sty
