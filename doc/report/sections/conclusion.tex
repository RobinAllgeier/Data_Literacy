\section{Discussion \& Conclusion}\label{sec:conclusion}

Our analysis of borrowing records from the Tübingen City Library reveals several behavioral patterns.
Most notably, late return behavior is not primarily driven by persistent user unreliability, but rather by lack of experience during early library engagement.

Late return probability declines sharply from approximately 13\% in first sessions to close to 8\% over the initial sessions, while extension requests rise from 50\% to over 65\% during the same period.
This diverging pattern indicates that users learn to navigate library procedures proactively rather than simply becoming more punctual.
Even among experienced users, late returns persist due to structural factors beyond user knowledge or behavior.

These findings offer practical insights for the Tübingen City Library.
Overdue risk is concentrated in early sessions, suggesting that targeted communication and reminder systems for new users may be more effective than uniform enforcement mechanisms.
For instance, automated reminders or welcome materials explaining borrowing procedures could be specifically targeted at users during their first visits.
The increasing use of extensions indicates that users value this system, and ensuring it remains accessible could further reduce involuntary late returns.

Beyond late returns, we observed that borrowing activity concentrates during daytime hours aligned with work and school schedules, while media-type preferences remain influential but become less predictive over sessions as experienced users expand their borrowing range.
These patterns reveal how users adapt their behavior to library operations over time.
Understanding these temporal patterns could inform staffing decisions and resource allocation during peak usage hours.
Additional analyses available in the accompanying repository examine factors such as media type, user categories, and seasonal patterns that influence late return behavior.

Several limitations constrain our analysis.
The dataset covers only 2019 onwards due to earlier system conventions, and user behavior can only be observed indirectly through transactions.
Contextual factors such as whether items are borrowed for oneself or others remain unobserved.

Nevertheless, this work demonstrates the value of systematically analyzing library circulation data to support informed operational decisions.
The session-based representation and preprocessing framework established here provides a foundation for further analysis, such as evaluating the effectiveness of reminder systems or examining policy changes over time.
By revealing learning effects and behavioral adaptation patterns, libraries can move from reactive enforcement toward proactive support systems.
