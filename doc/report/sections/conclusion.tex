\section{Discussion \& Conclusion}\label{sec:conclusion}

% Use this section to briefly summarize the entire text. Highlight limitations and problems, but also make clear statements where they are possible and supported by the analysis. 
Public libraries generate rich datasets that provide a detailed view of user borrowing behavior. However, leveraging these data requires extensive domain knowledge. Understanding library rules, operating hours, internal system conventions, and similar operational details is essential to interpret the data correctly. This work was based on a non-public and mostly unprocessed dataset from the Tübingen City Library. This setting presented both challenges and opportunities. 
A significant amount of the work was devoted to evaluating and cleaning up the data quality, which relied in parts on the library's insights to resolve inconsistencies and understand the structure of the data.

There are several limitations to consider. First, the dataset only includes borrowing records from 2019 onwards, as earlier data were collected using different system conventions. This restricts the temporal scope of the analysis and prevents the investigation of long-term trends. Second, user behavior can only be observed indirectly through recorded transactions. As the temporal analysis suggests, several sources of bias and external influences are not captured in the data. These include constraints on user schedules, social contexts and unobserved personal circumstances. Such factors limit the precision with which behavioral patterns can be inferred. For example, it remains unknown whether items are borrowed for the users themselves or for others (e.g., parents borrowing for their children), which can affect the interpretation of observed behavior.

Nevertheless, the analysis of user behavior revealed both aggregated usage patterns describing daily library use and individual-level behaviors. The results indicate that temporally consistent borrowing behavior is difficult for the majority of users, while a small subgroup nevertheless exhibits highly regular temporal routines. Furthermore, the analysis of media-type choices suggests that dominant preferences are relatively unstable in early sessions but become more consistent over time, indicating an initial exploration phase followed by more stable borrowing behavior.

Regarding the central issue of overdue behavior, the results suggest that late returns are not primarily driven by persistent user unreliability, but rather by a lack of experience and routine in early stages of library usage.
Over successive sessions, users exhibit learning effects. Overdue rates decline, while the use of the renewal system increases. This indicates that users adapt to institutional rules and develop strategies to manage borrowing deadlines more effectively over time. From a practical perspective, these findings imply that targeted communication and reminder systems for new users may be more effective than uniform enforcement mechanisms, as they address the phase in which overdue risk is highest.

Overall, the findings underline the value of systematically analyzing library data to support informed operational and policy decisions. Moreover, the session-based representation and preprocessing framework established in this work provides a basis for further analysis.
