\section{Discussion \& Conclusion}\label{sec:conclusion}

% Use this section to briefly summarize the entire text. Highlight limitations and problems, but also make clear statements where they are possible and supported by the analysis. 
Public libraries generate rich, multifaceted datasets that provide a detailed view of user borrowing behavior. However, leveraging these data requires extensive domain knowledge. Understanding library rules, operating hours, internal system conventions, and similar operational details is essential to interpret the data correctly. This work was based on a non-public and mostly unprocessed dataset from the Tübingen City Library. It goes beyond the internal statistics compiled by the library. This setting presented both challenges and opportunities. 
A significant amount of the work was devoted to evaluating and cleaning up the data quality, which relied in parts on the library's insights to resolve inconsistencies and understand the structure of the data.
There are several limitations to consider. First, the dataset only includes borrowing records from 2019 onwards, as earlier data were collected under different system conventions. This restricts the temporal scope of the analysis and prevents the investigation of longer-term trends. Second, user behavior can only be observed indirectly through recorded transactions. As the temporal analysis suggests, several sources of bias and external influences are not captured in the data, such as constraints on user schedules, social contexts, or unobserved personal circumstances, which limit the precision with which behavioral patterns can be inferred. For example, it remains unknown whether items are borrowed for the users themselves or for others (e.g., parents borrowing for their children), which can affect the interpretation of observed behavior.
Nevertheless, the analysis of user behavior revealed both aggregated usage patterns describing daily library use and individual-level behaviors. The results indicate that temporally consistent borrowing behavior is difficult for the majority of users, while a small subgroup nevertheless exhibits highly regular temporal routines.
In addition, the data show learning effects with repeated use. This reflects not only improved handling of late returns, but also the development of adaptive strategies that reduce late returns through active use of the renewal system.

Overall, the results also have practical relevance for the library. Understanding user behavior can inform staffing decisions, opening hours, and the design of reminder systems or targeted communication. Moreover, the session-based representation and preprocessing pipeline provides a foundation for further analyses on this dataset.