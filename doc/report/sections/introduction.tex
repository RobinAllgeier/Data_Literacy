\section{Introduction}\label{sec:intro}

% Motivate the problem, situation or topic you decided to work on. Describe why it matters (is it of societal, economic, scientific value?). Outline the rest of the paper (use references, e.g.~to \Cref{sec:methods}: What kind of data you are working with, how you analyse it, and what kind of conclusion you reached. The point of the introduction is to make the reader want to read the rest of the paper.

Public libraries play an important role in providing people with access to knowledge, education, and entertainment. However, they often face challenges related to the timely return of borrowed items. Late returns can lead to reduced availability of popular materials, diminished user satisfaction, who wait for the media, and increased administrative burdens for library staff.
Understanding which factors are associated with late returns is therefore very important. If libraries can identify borrowing characteristics or user patterns that predict a higher likelihood of delayed returns, they can adapt their policies such as loan duration limits, extension rules, or targeted reminders to improve service quality and resource availability.
This study analyzes borrowing records from a public library, to answer the question: \textit{What observable characteristics of a loan are associated with late returns?} Specifically, we investigate the relationship between late returns and factors such as loan duration, number of extensions, media type, user category, and temporal patterns (e.g., month or weekday of borrowing).

Our analysis reveals several clear patterns: \dots