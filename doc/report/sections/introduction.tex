\section{Introduction}\label{sec:intro}

% Motivate the problem, situation or topic you decided to work on. Describe why it matters (is it of societal, economic, scientific value?). Outline the rest of the paper (use references, e.g.~to \Cref{sec:methods}: What kind of data you are working with, how you analyse it, and what kind of conclusion you reached. The point of the introduction is to make the reader want to read the rest of the paper.

Public libraries play an important role in providing people with access to knowledge, education, and entertainment. To maintain efficient operations and ensure equitable resource availability, libraries must understand how their users interact with the collection. Analyzing borrowing behavior provides valuable insights into usage patterns, user preferences, and operational challenges, enabling data-driven decision-making to improve service quality.
Understanding borrowing behavior encompasses multiple dimensions: which types of media are most frequently borrowed, how different user groups interact with the library, how loan durations and extension patterns vary across media types and user categories, and temporal trends in borrowing activity. One particularly important aspect of borrowing behavior is the timely return of borrowed items. Late returns can lead to reduced availability of popular materials, diminished satisfaction among users waiting for specific items, and increased administrative burdens for library staff. Identifying factors associated with late returns enables libraries to adapt their policies, such as loan duration limits, extension rules, or targeted reminder systems—to improve resource circulation and user satisfaction.
This study analyzes borrowing records from a public library. The primary objective is to characterize borrowing behavior across multiple dimensions and to identify patterns that contribute to our understanding of library usage. Specifically, this study addresses the following questions: \textit{How do borrowing patterns vary across media types, user categories, and time periods? What observable characteristics of a loan are associated with late returns?} The analysis investigates temporal patterns, in borrowing activity, and examines the relationships between late returns and several factors.

Our analysis reveals several clear patterns: \dots