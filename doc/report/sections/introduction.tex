\section{Introduction}\label{sec:intro}

% Motivate the problem, situation or topic you decided to work on. Describe why it matters (is it of societal, economic, scientific value?). Outline the rest of the paper (use references, e.g.~to \Cref{sec:methods}: What kind of data you are working with, how you analyse it, and what kind of conclusion you reached. The point of the introduction is to make the reader want to read the rest of the paper.

Public libraries serve a critical role in democratic societies by providing equal access to knowledge, education, and cultural resources.
Understanding user borrowing behavior enables libraries to improve service quality and ensure materials remain available to all.
This study analyzes borrowing records from the Tübingen City Library, addressing key operational questions identified in collaboration with library staff to support evidence-based decision-making.
Specifically, we investigate late return patterns, user learning effects, and temporal borrowing regularities.

A central operational challenge is the timely return of borrowed materials.
Late returns disrupt circulation efficiency: when items remain overdue, they are marked as on-loan in the system but are neither available for other patrons nor counted as part of the accessible collection.
In severe cases, overdue items can escalate to multiple reminders and eventually legal action to recover replacement costs, straining administrative resources.
Studies of academic library circulation demonstrate that return behavior is shaped by multiple factors, including user characteristics, reminders, and institutional policies~\cite{library_fines_return}.

Beyond late returns, understanding broader borrowing patterns such as temporal regularities, learning effects, and media preferences enables data-driven policy decisions.
We develop a session-based analytical framework (\Cref{sec:methods}) that aggregates item-level records from a large-scale dataset into user visits, recognizing that patrons typically borrow multiple items during a single library visit.
Our analysis (\Cref{sec:results}) reveals distinct temporal usage patterns tied to operating hours and shows that late returns decline over initial sessions while extension use increases.
These findings can be interpreted in terms of user learning, behavioral adaptation, and their implications for library operations (\Cref{sec:conclusion}).
