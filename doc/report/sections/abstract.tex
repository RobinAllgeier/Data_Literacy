Late returns of library materials disrupt circulation efficiency and strain administrative resources. This study analyzes 2.4 million loan transactions from approximately 21,000 users at the Tübingen City Library (2019–2025) to understand borrowing behavior patterns and learning effects. We develop a session-based analytical framework that aggregates item-level records into user visits. Our analysis reveals that late return probability declines sharply during early sessions, while extension requests increase from 50\% to 65\%, indicating proactive learning rather than simple compliance. Borrowing activity exhibits strong temporal patterns aligned with operating hours, and media-type preferences remain influential but become less predictive with experience. These findings suggest that targeted communication systems for new users may be more effective than uniform enforcement mechanisms.