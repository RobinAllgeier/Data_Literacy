\section{Data and Methods}\label{sec:methods}

%In this section, describe \emph{what you did}. Roughly speaking, explain what data you worked with, how or from where it was collected, it's structure and size. Explain your analysis, and any specific choices you made in it. Depending on the nature of your project, you may focus more or less on certain aspects. If you collected data yourself, explain the collection process in detail. If you downloaded data from the net, show an exploratory analysis that builds intuition for the data, and shows that you know the data well. If you are doing a custom analysis, explain how it works and why it is the right choice. If you are using a standard tool, it may still help to briefly outline it. Cite relevant works. You can use the \verb|\citep| and \verb|\citet| commands for this purpose \citep{mackay2003information}.

% This is the template for a figure from the original ICML submission pack. In lecture 10 we will discuss plotting in detail.
% Refer to this lecture on how to include figures in this text.
% 
% \begin{figure}[ht]
% \vskip 0.2in
% \begin{center}
% \centerline{\includegraphics[width=\columnwidth]{icml_numpapers}}
% \caption{Historical locations and number of accepted papers for International
% Machine Learning Conferences (ICML 1993 -- ICML 2008) and International
% Workshops on Machine Learning (ML 1988 -- ML 1992). At the time this figure was
% produced, the number of accepted papers for ICML 2008 was unknown and instead
% estimated.}
% \label{icml-historical}
% \end{center}
% \vskip -0.2in
% \end{figure}

\subsection{Data Collection and Description}
This study investigates the borrowing behavior of users at the Tübingen City Library, as well as the frequency of late returns. The dataset was provided directly by the Tübingen City Library and covers borrowing records from 2019 to 2025. It encompasses over 2.4 million individual loan transactions and includes the following key variables: borrowing and return timestamps, media types, user categories, number of extensions, anonymized user identification numbers, and late return flags. Each record documents a complete transaction from initial borrowing to return.
Domain knowledge was gathered through a personal interview with a staff member of the Tübingen City Library, providing contextual insights into the data collection processes, library operations, and data quality practices. Additional information was obtained through ongoing email communication with the library staff, which helped to clarify data inconsistencies and provide domain expertise for informed analysis decisions.

\subsection{Data Quality Assessment: Sanity Checks}
Prior to conducting any analysis, a comprehensive set of data quality checks was performed to validate internal consistency and identify potential data quality issues. These sanity checks were designed to detect anomalies without filtering or removing data at first, and gather information, regarding cleaning the data to ensure robust analysis afterwards. The following specific checks were implemented:

\subsubsection{Missing Values}
First of all, the overall of missing values in the dataset was assessed. The proportion of missing values in each column was calculated to identify any fields with significant gaps that could impact the analysis. Depending on the extent and nature of the missing valus, they were excluded from certain analyses.
Return timestamps are missing in roughly two percent of rows, which is expected for currently borrowed and unreturned or lost items and therefore retained. The same rate applies to the derived duration column because it depends on return timestamps. A missing rate of about 6.7\% was found in the user ID colum, which will be handled accordingly in the analysis. All other colums, which are relevant for our analyses, have a missing rate below 0.1\% and are therefore considered complete.

\subsubsection{Timestamp and Duration Consistency}
The integrity of temporal data was verified by identifying logical inconsistencies, such as instances where return dates preceded borrowing dates. These checks are critical since all downstream analyses depend on accurate temporal information. Further checks on the duration value revealed that it is calculated correctly from the timestamps and does not contain any inconsistencies.

\subsubsection{Late Return Consistency}
The consistency between late return flags and the reported number of days late was validated. No Entries were identified where the late flag indicated ``not late'' but the days-late counter was positive, and conversely, records marked as late but with zero or missing days-late values. Additionally, implausible values in the extensions column (negative extensions or more than six extensions) were checked. There are a view cases where the number of extentions is higher than six, which should not be possible according to the library's policies. But because of the low number of affected rows (0.003\%), these where not further investigated.

\subsubsection{Duplicate Analysis}
Various forms of duplicates were examined. There are no exact duplicates in the dataset. However, some entries show the the exact same borrowing timestamp. These can be explanied by the fact, that users can borrow at multiple stations at the same time. Also it is possilbe for user to borrow multiple items in one transaction, resulting in identical borrowing timestamps for different items, but all these occurances limit to the same user ID and to a maximum of seven items/entries per transaction. Only one case was found where a diffetent user borrowed and returned at the exact same times, which is considered a rare but possible event and therefore retained.

\subsection{Analysis Approach}
For the analysis of borrowing behavior and late returns, descriptive statistical methods were employed to identify patterns and trends in user behavior. These methods were chosen to provide a comprehensive understanding of borrowing dynamics and to isolate factors associated with late returns. The analysis focuses on temporal patterns, user segments, media types, and their relationships with return timeliness.
