\section{Data and Methods}\label{sec:methods}

%In this section, describe \emph{what you did}. Roughly speaking, explain what data you worked with, how or from where it was collected, it's structure and size. Explain your analysis, and any specific choices you made in it. Depending on the nature of your project, you may focus more or less on certain aspects. If you collected data yourself, explain the collection process in detail. If you downloaded data from the net, show an exploratory analysis that builds intuition for the data, and shows that you know the data well. If you are doing a custom analysis, explain how it works and why it is the right choice. If you are using a standard tool, it may still help to briefly outline it. Cite relevant works. You can use the \verb|\citep| and \verb|\citet| commands for this purpose \citep{mackay2003information}.

% This is the template for a figure from the original ICML submission pack. In lecture 10 we will discuss plotting in detail.
% Refer to this lecture on how to include figures in this text.
% 
% \begin{figure}[ht]
% \vskip 0.2in
% \begin{center}
% \centerline{\includegraphics[width=\columnwidth]{icml_numpapers}}
% \caption{Historical locations and number of accepted papers for International
% Machine Learning Conferences (ICML 1993 -- ICML 2008) and International
% Workshops on Machine Learning (ML 1988 -- ML 1992). At the time this figure was
% produced, the number of accepted papers for ICML 2008 was unknown and instead
% estimated.}
% \label{icml-historical}
% \end{center}
% \vskip -0.2in
% \end{figure}

\subsection{Data Collection and Description}
This study investigates the borrowing behavior of users at the Tübingen City Library, as well as the frequency of late returns. The dataset was provided directly by the Tübingen City Library and covers borrowing records from 2019 to 2025. It encompasses over 2.4 million individual loan transactions and includes the following key variables: borrowing and return timestamps, 21 different media types, user categories, number of extensions, anonymized user identification numbers, as well as late return flags. Each record documents a complete transaction from initial borrowing to return.
Domain knowledge was gathered through a personal interview with a staff member of the Tübingen City Library, providing contextual insights into the data collection processes, library operations, and data quality practices. Additional information was obtained through ongoing email communication with the library staff, which helped to clarify data inconsistencies and provide domain expertise for informed analysis decisions.

\subsection{Data Quality Assessment: Sanity Checks}
Proir to conducting any analysis, a comprehensive set of data quality checks were preformed to validate internal consistency and identify potential data quality issues. The following checks were implemented:
\textbf{Missing Values:} The proportion of missing values was assessed for each column. Retrun timestamps were missing in apporximately 2\% of rows, which is expected for currently borrowed or lost items and therefore retained. User IDs showed a missing rate of 6.7\%, which was handled accordingly in subsequent analyses. All other relevant columns had such low missing rates, that they were considered negligible. 
\textbf{Temporal Consistency:} The integrity of temporal data was verified by checking for logical inconsistencies, such as rerurn dates preceding borrowing dates and a correct calculation of duration values from the timestamps. 
\textbf{Late Return Consistency:} The consistency between late return flags and the number of days late was validated. Additionally, implausible values in the extensions column were checked. Only a few cases with more than six extentions were identified (0.003\%), which  but were not further investugated due to their negligible impact.
\textbf{Dublicate Analysis:} Various forms of duplicates were examined. No exact duplicates were found, but identical borrowing timestamps occurred when useres borrowed multiple items in one transaction, with a maximum of seven items per transaction. These were retained as they represent valid borrowing behavior.

\subsection{Data cleaning}
Based on the result of the data quality assessment, the data was cleaned, based on four rules. The following entries were removed from the dataset:
\begin{itemize}
    \item Entries with missing retuern timestamps, as they represent currently borrowed or lost items.
    \item Entries with a borrowing time longer than the 196 days, because they extend the maximum borrowing period of 28 days and six extentions, which are considered implausible.
    \item Borrowings taken by library`s staff members and system entries, which do not reflect typical user behavior.
    \item For some analyses, all media types other than books, because they have different borrowing duration limits and would distort the results.
\end{itemize}
These cleaning steps ensured that the dataset used for analysis was both reliable and relevant, thereby enhancing the validity of the findings.

\subsection{Analysis Approach}
The analysis focuses on three key aspects of user behavior at the library:

\subsubsection{Temporal Distribution of Library Visits}
First, we investigate the distribution of borrowing events across different times of the day and days of the week. Library opening hours (09:00–20:00) were distinguished from closed hours to identify peak activity periods. We aggregated the number of unique users per 30-minute time bin separately for weekdays and weekends, revealing patterns of user traffic throughout the day. 

\subsubsection{User Visit Regularity}
Second, we examine whether individual users visit the library at predictable times. 

\subsubsection{late and learning rate}
Third, we analyze the frequency of late returns and how users adapt their borrowing behavior over time. The late return rate is calculated as the proportion of borrowings returned after the due date. Additionally, we assess whether users learn from their past late returns by examining changes in their borrowing durations in subsequent transactions. This involves tracking individual users' borrowing histories and analyzing trends in their return behaviors over time. For these analyses borrowing entries where grouped in sessions, defined as sequence of borrowings by the same user performed within 24 hours. If a medium from one session is returned late, the whole session is considered late. This approach captures the idea that users may adjust their behavior based on their overall experience during a visit, rather than on individual items alone.
Also the number of extentions is analyzed in relation to late returns.