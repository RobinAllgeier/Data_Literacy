\section{Results}\label{sec:results}

In this section outline your results. At this point, you are just stating the outcome of your analysis. You can highlight important aspects (``we observe a significantly higher value of $x$ over $y$''), but leave interpretation and opinion to the next section. This section absoultely \emph{must} include at least two figures.



\begin{figure}[ht]
    \centering
    \includegraphics{figures/plot_1_learning_curve.pdf}
    \caption{
        Session-based late-return (\textcolor{pn_orange}{%
            \tikz\draw[line width=1.2pt] (0,0) -- (0.9em,0)
            node[midway, circle, fill=pn_orange, inner sep=1.2pt] {};
        } left axis) 
        and extension 
        (\textcolor{TUblue}{%
            \tikz\draw[line width=1.2pt] (0,0) -- (0.9em,0)
            node[midway, circle, fill=TUblue, inner sep=1.2pt] {};
        } right axis) probabilities
        across the user session index.
        Shaded bands denote user-level bootstrap confidence intervals,
        while thin lines show raw estimates and thicker lines rolling averages.
    }
    \label{fig:learning-curve}
\end{figure}




\begin{figure}[ht]
    \centering
    \includegraphics{figures/plot_2_library_visit_regularities.pdf}
    \caption{
        Average library usage by time of day on a 24-hour clock.
        Mean session counts per time bin are shown for Tue--Fri
        (\textcolor{TUblue}{%
            \tikz[baseline]{%
                \draw[fill=TUblue!25, draw=TUblue, line width=0.6pt]
                (0,0) rectangle (0.55em,0.55em);
            }
        }\!)
        and Saturday
        (\textcolor{pn_orange}{%
            \tikz[baseline]{%
                \draw[fill=pn_orange!25, draw=pn_orange, line width=0.6pt]
                (0,0) rectangle (0.55em,0.55em);
            }
        }\!).
        Nighttime hours are visually compressed to reflect library closing times, which are indicated by grey radial bands.
    }
    \label{fig:visit-clock}
\end{figure}




