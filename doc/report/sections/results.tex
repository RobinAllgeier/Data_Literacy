\section{Results}\label{sec:results}

%In this section outline your results. At this point, you are just stating the outcome of your analysis. You can highlight important aspects (``we observe a significantly higher value of $x$ over $y$''), but leave interpretation and opinion to the next section. This section absoultely \emph{must} include at least two figures.

Borrowing activity at the Tübingen City Library exhibits strong temporal patterns tied to operating hours and user schedules.
\autoref{fig:visit-clock} displays average session counts aggregated into half-hour bins, separately for weekdays (Tuesday–Friday) and Saturdays (Monday/Sunday closed).
Each bin represents the mean number of sessions occurring within that time window, averaged across all recorded days.

\begin{figure}[ht]
    \centering
    \includegraphics{figures/plot_1_clock_plot.pdf}
    \caption{
        Average library usage by time of day on a 24-hour clock.
        Mean session counts per time bin are shown for Tue--Fri
        (\textcolor{TUblue}{%
            \tikz[baseline]{%
                \draw[fill=TUblue!25, draw=TUblue, line width=0.6pt]
                (0,0) rectangle (0.55em,0.55em);
            }
        }\!)
        and Saturday
        (\textcolor{pn_orange}{%
            \tikz[baseline]{%
                \draw[fill=pn_orange!25, draw=pn_orange, line width=0.6pt]
                (0,0) rectangle (0.55em,0.55em);
            }
        }\!).
        Nighttime hours are visually compressed to reflect library closing times, which are indicated by grey radial bands.
    }
    \label{fig:visit-clock}
\end{figure}

Usage concentrates heavily during daytime hours, with clear peaks from late morning through afternoon.
Weekdays show sustained high volume from opening through early afternoon, with a noticeable dip around midday consistent with lunch breaks.
Activity rises again in the afternoon, aligning with work and school schedules.
Saturday exhibits a later start and more compressed activity, suggesting leisure-driven rather than routine use.
These aggregate patterns reveal shared daily rhythms shaped by operational hours and typical schedules, though individuals vary considerably in their temporal regularity.

Among users with at least 10 sessions, the mean probability of returning on the same weekday is 40.4\%, indicating moderate day-of-week preferences, with some showing stronger consistency.
Visit times within a day are more dispersed, with a mean standard deviation of 2.1 hours across users, suggesting most do not restrict themselves to narrow time windows.
However, approximately 3\% exhibit high temporal precision, with standard deviations below 1 hour, enabling identification of preferred time slots based on their mean visit times.

Beyond these temporal patterns, user behavior evolves with repeated library engagement, revealing learning effects in late returns and loan extensions.
\autoref{fig:learning-curve} plots both probabilities against the session index, tracking users from their first visit through their 25th session.
Late-return probability shows a sharp decline over the first nine sessions, dropping from approximately 13\% to below 10\%, indicating rapid acquisition of library procedures such as due dates and reminder systems.
Beyond session 10, the curve stabilizes into a relatively flat plateau around 8–9\%, suggesting users have transitioned from active learning into stable behavioral routines.

\begin{figure}[ht]
    \centering
    \includegraphics{figures/plot_2_learning_curve.pdf}
    \caption{
        Session-based late-return (\textcolor{pn_orange}{%
            \tikz\draw[line width=1.2pt] (0,0) -- (0.9em,0)
            node[midway, circle, fill=pn_orange, inner sep=1.2pt] {};
        } left axis)
        and extension
        (\textcolor{TUblue}{%
            \tikz\draw[line width=1.2pt] (0,0) -- (0.9em,0)
            node[midway, circle, fill=TUblue, inner sep=1.2pt] {};
        } right axis) probabilities
        across the user session index.
        Shaded bands denote user-level bootstrap confidence intervals,
        while thin lines show raw estimates and thicker lines rolling averages.
    }
    \label{fig:learning-curve}
\end{figure}

Extension probability exhibits the inverse pattern, rising from approximately 50\% to over 65\% during the same experience window.
This increase reveals a shift in user strategy: rather than simply becoming more punctual, experienced users proactively request extensions to avoid lateness.
The diverging trajectories suggest that learning manifests not as perfect compliance, but as adaptive behavior within library constraints.
Notably, late returns never approach zero even among highly experienced users, indicating that some lateness stems from structural factors such as fixed loan limits or schedule conflicts, independent of user familiarity with procedures.

\begin{figure}[ht]
    \centering
    \includegraphics{figures/plot_4_media_type_stickiness.pdf}
    \caption{
        Consistency with early dominant media type across the user session index for
        $k_0=1$
        (\textcolor{pn_orange}{%
            \tikz\draw[line width=1.2pt] (0,0) -- (0.9em,0)
            node[midway, circle, fill=pn_orange, inner sep=1.2pt] {};
        }\!),
        $k_0=5$
        (\textcolor{TUblue}{%
            \tikz\draw[line width=1.2pt] (0,0) -- (0.9em,0)
            node[midway, circle, fill=TUblue, inner sep=1.2pt] {};
        }\!),
        and $k_0=10$
        (\textcolor{TUred}{%
            \tikz\draw[line width=1.2pt] (0,0) -- (0.9em,0)
            node[midway, circle, fill=TUred, inner sep=1.2pt] {};
        }\!).
        For each $k_0$, the curve starts at session $k_0+1$.
        Shaded bands denote user-level bootstrap confidence intervals,
        while thin lines show raw estimates and thicker lines rolling averages.
    }
    \label{fig:media-stickiness}
\end{figure}


In addition to timing and return behavior, we analyze the consistency of media-type choices over repeated sessions at the user level.
\autoref{fig:media-stickiness} shows, for $k_0 \in \{1,5,10\}$, the share of sessions whose dominant media type matches
the user’s early preferred type, plotted against the session index. Across all three curves, consistency decreases with
increasing session index, with the strongest decline for smaller $k_0$.
For $k_0=1$, users match their first-session dominant media type in about 64\% of second sessions.
By session 25, this value drops to roughly 50\%, a decrease of about 14 percentage points.
For $k_0=10$, the decrease is about 7 percentage points, and the curve remains close to 60\% in later sessions.
This is consistent with the fact that choosing a larger $k_0$ defines the early preferred type from more observations, making it more robust and less sensitive
to variability in the first sessions.
Overall, the curves flatten over time, indicating that most change in dominant media type occurs in the early sessions,
while behavior becomes more stable at higher session indices.
