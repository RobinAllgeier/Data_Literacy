\section{Results}\label{sec:results}

%In this section outline your results. At this point, you are just stating the outcome of your analysis. You can highlight important aspects (``we observe a significantly higher value of $x$ over $y$''), but leave interpretation and opinion to the next section. This section absoultely \emph{must} include at least two figures.

Borrowing activity at the Tübingen City Library exhibits strong temporal patterns tied to operating hours and user schedules. 
\autoref{fig:visit-clock} displays average session counts aggregated into half-hour bins, separately for weekdays (Tuesday–Friday) and Saturdays (Monday/Sunday closed). 
Each bin represents the mean number of sessions occurring within that time window, averaged across all recorded days.

\begin{figure}[H]
    \centering
    \includegraphics{figures/plot_2_library_visit_regularities.pdf}
    \caption{
        Average library usage by time of day on a 24-hour clock.
        Mean session counts per time bin are shown for Tue--Fri
        (\textcolor{TUblue}{%
            \tikz[baseline]{%
                \draw[fill=TUblue!25, draw=TUblue, line width=0.6pt]
                (0,0) rectangle (0.55em,0.55em);
            }
        }\!)
        and Saturday
        (\textcolor{pn_orange}{%
            \tikz[baseline]{%
                \draw[fill=pn_orange!25, draw=pn_orange, line width=0.6pt]
                (0,0) rectangle (0.55em,0.55em);
            }
        }\!).
        Nighttime hours are visually compressed to reflect library closing times, which are indicated by grey radial bands.
    }
    \label{fig:visit-clock}
\end{figure}

Usage concentrates heavily during daytime hours, with clear peaks from late morning through afternoon. 
Weekdays show sustained high volume from opening through early afternoon, with a noticeable dip around midday, consistent with lunch breaks. 
Activity then rises again in the afternoon, aligning with work and school schedules. 
Saturday exhibits a later start and more compressed activity, suggesting leisure-driven visits instead of routine use. 
The plot reveals that library visits follow shared daily rhythms rather than uniform distribution, shaped by operational hours and typical schedules.

Beyond these temporal patterns, user behavior evolves with repeated library engagement, revealing learning effects in late returns and loan extensions.
\autoref{fig:learning-curve} plots both probabilities against the session index, tracking users from their first visit through their 25th session. 
Late-return probability shows a sharp decline over the first nine sessions, dropping from approximately 13\% to below 10\%, indicating rapid acquisition of library procedures such as due dates and reminder systems. 
Beyond session 10, the curve stabilizes into a relatively flat plateau around 8–9\%, suggesting users have transitioned from active learning into stable behavioral routines.

\begin{figure}[H]
    \centering
    \includegraphics{figures/plot_1_learning_curve.pdf}
    \caption{
        Session-based late-return (\textcolor{pn_orange}{%
            \tikz\draw[line width=1.2pt] (0,0) -- (0.9em,0)
            node[midway, circle, fill=pn_orange, inner sep=1.2pt] {};
        } left axis) 
        and extension 
        (\textcolor{TUblue}{%
            \tikz\draw[line width=1.2pt] (0,0) -- (0.9em,0)
            node[midway, circle, fill=TUblue, inner sep=1.2pt] {};
        } right axis) probabilities
        across the user session index.
        Shaded bands denote user-level bootstrap confidence intervals,
        while thin lines show raw estimates and thicker lines rolling averages.
    }
    \label{fig:learning-curve}
\end{figure}

Extension probability exhibits the inverse pattern, rising from approximately 50\% to over 65\% during the same experience window. 
This increase reveals a shift in user strategy: rather than simply becoming more punctual, experienced users proactively request extensions to avoid lateness. 
The diverging trajectories suggest that learning manifests not as perfect compliance, but as adaptive behavior within library constraints. 
Notably, late returns never approach zero even among highly experienced users, indicating that some lateness stems from structural factors such as fixed loan limits or schedule conflicts, independent of user familiarity with procedures.






